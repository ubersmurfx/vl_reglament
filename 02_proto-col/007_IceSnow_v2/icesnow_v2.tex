\pdfoutput=1
\documentclass[12pt]{article}
\usepackage[margin=2.5cm, top=1cm]{geometry}

% Required packages
\usepackage{amsmath,amsfonts,amsthm,amssymb,mathrsfs} % Mathematical typesetting and symbols
\usepackage{enumerate} % Custom enumeration labels
\usepackage{graphicx} % Figure inclusion
\usepackage{fancyhdr} % Перемещено сюда
\usepackage{hyperref} % Hyperlinks for citations, references, and URLs
\usepackage[numbers,sort&compress]{natbib} % Citation styling
\usepackage{subcaption} % Captions for subfigures
\usepackage{xcolor} % Text color
\usepackage[utf8x]{inputenc} % кодировка
\usepackage[english,russian]{babel} % добавление русского языка
\usepackage{cmap} % отображение юникод символов в pdf файле
\usepackage{eso-pic}
\usepackage{tikz}
\usepackage{ulem}
\usepackage{tabularx}
\usepackage{array} % Для выравнивания по правому краю
\usepackage{setspace}
\usepackage{xcolor}
\usepackage{booktabs}
% competition.tex
\usepackage{eso-pic}
\usepackage{xcolor}

\AddToShipoutPictureBG{%
    \ifnum\value{page}=1
        \put(4cm,26cm){\parbox[b][\paperheight]{\paperwidth}{%
        \fontsize{14}{14}\selectfont
        \color{gray!75}
        «Кубок РТК Высшая Лига: Финал, 26-28 ноября 2025 г.\par}}
    \fi
}

\usetikzlibrary{decorations.pathreplacing}

% Figure and table numbering by section
\counterwithin{figure}{section}
\counterwithin{table}{section}
\newenvironment{example}[1][]{\begin{ex}[#1]}{\hfill$\blacksquare$\end{ex}} % Add a black box to the end automatically, and accept an optional title argument
\newtheorem{ex}{Example}[section] % Examples are implemented as a type of Theorem

% Overriding abbrvnat BibTeX template to use the plainurl template's DOI hyperlinks
\newcommand*{\doi}[1]{\href{https://doi.org/#1}{\tt doi:~#1}}
\newcommand{\firstArma}{110}
\newcommand{\secondArma}{110}
\newcommand{\thirdArma}{110}
\newcommand{\fourthArma}{110}
\newcommand{\MasterHoming}{60}

%==============================================================================
\AddToShipoutPictureBG*{
    \ifnum\value{page}=1
        \put(2cm,26cm){\includegraphics[width=1.5cm]{figures/cup.jpg}}
    \fi
}

% background
\date{} % Добавлено для удаления даты
\title{Протокол для очного этапа \\ Хакатона:«Ледяной вызов»}

%==============================================================================
\begin{document}

\maketitle
\thispagestyle{empty}
\vspace*{-1.5cm}

\raisebox{1cm}{\begin{minipage}[t]{1\textwidth} % 0.8\textwidth - регулирует ширину minipage
    \begin{spacing}{2} % Устанавливает межстрочный интервал в 1.5 раза
        \textbf{Команда:} \uline{\hspace*{\fill}} \\
        \textbf{Участник:} \uline{\hspace*{10cm}} / \uline{\hspace*{\fill}} \\
        \textbf{Судья:} \uline{\hspace*{10.7cm}} / \uline{\hspace*{\fill}} 
    \end{spacing}
    \end{minipage}}


\vspace*{0.25cm}
\setlength{\extrarowheight}{12pt}
\begin{tabularx}{\textwidth}{>{\raggedright}X c *{4}{>{\centering\arraybackslash}p{1.3cm}}}
    \toprule
    \textbf{Задание} & \textbf{Баллы} & \textbf{S1} & \textbf{S2} & \textbf{S3} & \textbf{S4} \\
    \midrule
    Очищена арматура  №1 от снега  & \firstArma & $\square$ & $\square$ & $\square$ & $\square$ \\
    Очищена арматура  №2 от снега  & \secondArma & $\square$ & $\square$ & $\square$ & $\square$ \\
    Очищена арматура  №3 от снега  & \thirdArma & $\square$ & $\square$ & $\square$ & $\square$ \\
    Очищена арматура  №4 от снега  & \fourthArma & $\square$ & $\square$ & $\square$ & $\square$ \\
    Выполнен возврат робота на стартовую ячейку & \MasterHoming & $\square$ & $\square$ & $\square$ & $\square$ \\
    \midrule
    \textbf{Баллы за попытку} & & \rule{1cm}{0.5pt} & \rule{1cm}{0.5pt} & \rule{1cm}{0.5pt} & \rule{1cm}{0.5pt} \\
    \midrule
    \textbf{Лучший результат (S)} & \multicolumn{5}{c}{\rule{2cm}{0.5pt}} \\
    \midrule
    \textbf{Оставшееся время (T)} & \multicolumn{5}{c}{\rule{2cm}{0.5pt} сек} \\
    \midrule
    \textbf{Бонус за время (B)} & \multicolumn{5}{c}{$(600 - T) \times \dfrac{S}{500} \times \dfrac{1}{6}$} \\
    \bottomrule
    \textbf{Итоговый балл (S) + (B)} & \multicolumn{5}{c}{\rule{2cm}{0.5pt}} \\
\end{tabularx}

\end{document}