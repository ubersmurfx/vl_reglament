\pdfoutput=1
\documentclass[12pt]{article}
\usepackage[margin=2.5cm, top=1cm]{geometry}

% Required packages
\usepackage{amsmath,amsfonts,amsthm,amssymb,mathrsfs} % Mathematical typesetting and symbols
\usepackage{enumerate} % Custom enumeration labels
\usepackage{graphicx} % Figure inclusion
\usepackage{fancyhdr} % Перемещено сюда
\usepackage{hyperref} % Hyperlinks for citations, references, and URLs
\usepackage[numbers,sort&compress]{natbib} % Citation styling
\usepackage{subcaption} % Captions for subfigures
\usepackage{xcolor} % Text color
\usepackage[utf8x]{inputenc} % кодировка
\usepackage[english,russian]{babel} % добавление русского языка
\usepackage{cmap} % отображение юникод символов в pdf файле
\usepackage{eso-pic}
\usepackage{tikz}
\usepackage{ulem}
\usepackage{tabularx}
\usepackage{array} % Для выравнивания по правому краю
\usepackage{setspace}
\usepackage{xcolor}

\usetikzlibrary{decorations.pathreplacing}

% Figure and table numbering by section
\counterwithin{figure}{section}
\counterwithin{table}{section}
\newenvironment{example}[1][]{\begin{ex}[#1]}{\hfill$\blacksquare$\end{ex}} % Add a black box to the end automatically, and accept an optional title argument
\newtheorem{ex}{Example}[section] % Examples are implemented as a type of Theorem

% Overriding abbrvnat BibTeX template to use the plainurl template's DOI hyperlinks
\newcommand*{\doi}[1]{\href{https://doi.org/#1}{\tt doi:~#1}}
\newcommand{\penaltyBusPoints}{\color{red}-50}
\newcommand{\boardingPassangersFirstZone}{115}
\newcommand{\boardingPassangersSecondZone}{135}
\newcommand{\passangersDelievery}{160}
\newcommand{\complianceTrafficRules}{100}

%==============================================================================
\AddToShipoutPictureBG*{
    \ifnum\value{page}=1
        \put(2cm,26cm){\includegraphics[width=1.5cm]{figures/cup.jpg}}
    \fi
}

\AddToShipoutPictureBG{
    \ifnum\value{page}=1
        \put(4cm,26cm){\parbox[b][\paperheight]{\paperwidth}{%
        \fontsize{14}{14}\selectfont % Уменьшаем размер шрифта
        \color{gray!75} % Делаем текст полупрозрачным (50% серого)
        «Высшая Лига: ВШЭ, 24-25 мая 2025 г.\par}}
    \fi
}

% background
\date{} % Добавлено для удаления даты
\title{Протокол для очного этапа \\ Хакатона:«Движение по городу»}

%==============================================================================
\begin{document}

\maketitle
\thispagestyle{empty}
\vspace*{-2cm}

\raisebox{1cm}{\begin{minipage}[t]{1\textwidth} % 0.8\textwidth - регулирует ширину minipage
    \begin{spacing}{1.5} % Устанавливает межстрочный интервал в 1.5 раза
        \textbf{Команда:} \uline{\hspace*{\fill}} \\
        \textbf{Участник:} \uline{\hspace*{10cm}} / \uline{\hspace*{\fill}} \\
        \textbf{Судья:} \uline{\hspace*{10.7cm}} / \uline{\hspace*{\fill}} 
    \end{spacing}
    \end{minipage}}

\renewcommand{\arraystretch}{1}
\vspace*{0.25cm}
\begin{tabularx}{\textwidth}{|X|c|p{2cm}|}
    \hline
    \textbf{Задание} & \textbf{Баллы} & \textbf{Результат}\\
    \hline
    Выполнена посадка первой группы пассажиров & \boardingPassangersFirstZone &  \\
    \hline
    Выполнена посадка второй группы пассажиров & \boardingPassangersSecondZone & \\
    \hline
    Пассажиры доставлены в зону высадки & \passangersDelievery & \\
    \hline 
    Соблюдено правило движения только по полосам & \complianceTrafficRules & \\
    \hline
    Вмешательство без штрафа (1) &  & \\
    \hline
    \textbf{Штрафы} & \textbf{Баллы} & \textbf{Результат} \\
    \hline
    Столкновение с общественным транспортом & \penaltyBusPoints &  \\
    \hline
\end{tabularx}
\vspace*{0.25cm}

\vspace*{0.25cm}
\begin{tabularx}{\textwidth}{|X|c|p{2cm}|}
    \hline
    \textbf{Задание} & \textbf{Баллы} & \textbf{Результат}\\
    \hline
    Выполнена посадка первой группы пассажиров & \boardingPassangersFirstZone &  \\
    \hline
    Выполнена посадка второй группы пассажиров & \boardingPassangersSecondZone & \\
    \hline
    Пассажиры доставлены в зону высадки & \passangersDelievery & \\
    \hline 
    Соблюдено правило движения только по полосам & \complianceTrafficRules & \\
    \hline
    Вмешательство без штрафа (2) &  & \\
    \hline
    \textbf{Штрафы} & \textbf{Баллы} & \textbf{Результат} \\
    \hline
    Столкновение с общественным транспортом & \penaltyBusPoints &  \\
    \hline
\end{tabularx}
\vspace*{0.25cm}

\vspace*{0.25cm}
\begin{tabularx}{\textwidth}{|X|c|p{2cm}|}
    \hline
    \textbf{Задание} & \textbf{Баллы} & \textbf{Результат}\\
    \hline
    Выполнена посадка первой группы пассажиров & \boardingPassangersFirstZone &  \\
    \hline
    Выполнена посадка второй группы пассажиров & \boardingPassangersSecondZone & \\
    \hline
    Пассажиры доставлены в зону высадки & \passangersDelievery & \\
    \hline 
    Соблюдено правило движения только по полосам & \complianceTrafficRules & \\
    \hline
    \textbf{Штрафы} & \textbf{Баллы} & \textbf{Результат} \\
    \hline
    Столкновение с общественным транспортом & \penaltyBusPoints &  \\
    \hline
\end{tabularx}
\vspace*{1cm}

\raisebox{1cm}{\begin{minipage}[t]{1\textwidth} % 0.8\textwidth - регулирует ширину minipage
    \begin{spacing}{1} % Устанавливает межстрочный интервал в 1.5 раза
        \textbf{Итоговое количество баллов:} \uline{\hspace*{4cm}}  \textbf{Оставшиеся время:} \uline{\hspace*{2cm}} 
    \end{spacing}
    \end{minipage}}
\end{document}