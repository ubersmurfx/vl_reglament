\pdfoutput=1
\documentclass[12pt]{article}
\usepackage[margin=2.5cm, top=2cm]{geometry}

% Required packages
\usepackage{amsmath,amsfonts,amsthm,amssymb,mathrsfs} % Mathematical typesetting and symbols
\usepackage{enumerate} % Custom enumeration labels
\usepackage{fancyhdr} % Custom header and footer
\usepackage{graphicx} % Figure inclusion
\usepackage{hyperref} % Hyperlinks for citations, references, and URLs
\usepackage[numbers,sort&compress]{natbib} % Citation styling
\usepackage{subcaption} % Captions for subfigures
\usepackage{xcolor} % Text color
\usepackage[utf8x]{inputenc} % кодировка
\usepackage[english,russian]{babel} % добавление русского языка
\usepackage{cmap} % отображение юникод символов в pdf файле
\usepackage{eso-pic}


% Figure and table numbering by section
\counterwithin{figure}{section}
\counterwithin{table}{section}
\newenvironment{example}[1][]{\begin{ex}[#1]}{\hfill$\blacksquare$\end{ex}} % Add a black box to the end automatically, and accept an optional title argument
\newtheorem{ex}{Example}[section] % Examples are implemented as a type of Theorem

% Course Definitions
\newcommand{\currentdate}{29.04.2025}
\newcommand{\courselong}{Редакция от }


% Overriding abbrvnat BibTeX template to use the plainurl template's DOI hyperlinks
\newcommand*{\doi}[1]{\href{https://doi.org/#1}{\tt doi:~#1}}

%==============================================================================


% background
\title{Регламент}
\date{Высшая Лига - 2025\footnote{Молодежные соревнования роботов, \href{https://cup.rtc.ru}{https://cup.rtc.ru}}}
\author{\courselong\\\currentdate}

\AddToShipoutPictureBG*{
    \ifnum\value{page}=1
        \put(2cm,25cm){\includegraphics[width=1cm]{figures/rtc.png}}
    \fi
}

%==============================================================================

\begin{document}

\maketitle \hspace*{0.5cm} 
\textbf{Организатор соревнований} ГНЦ РФ Центральный научно-исследовательский и опытно-конструкторский институт робототехники и технической кибернетики – центром компетенций в области исследований и создания средств робототехники, технической кибернетики и мехатроники, г. Санкт-Петербург (далее Организатор).
%==============================================================================
\newpage
\section{Общие положения}
\label{sec:intro}

%%%
\subsection{Цель} \hspace*{1cm}
Развитие научного потенциала, выявление талантливой молодежи и ее подготовка к решению задач, моделирующих реальные производственные и технологические задачи, путем создания автономных или с определенной степенью автономности роботов в рамках сценариев соревнований.

%%%
\subsection{Задачи} \hspace*{1cm}
Привлечение учащихся старших классов лицеев и техникумов, студентов ВУЗов и СУЗов (далее - участники) к научно-техническому творчеству в области робототехники и программирования.
\\\hspace*{1cm}
Формирование у участников новых знаний и умений в области применения алгоритмов технического зрения и навигации, супервизорного управления, создания автономных роботов.

%%%
\subsection{Суть соревнований} \hspace*{1cm}
Это соревнование роботов на специальном полигоне, где расположены задания для реализации алгоритмов технического зрения и навигации роботов в автономном режиме.
\\\hspace*{1cm}
В ходе соревнования роботы должны выполнить миссию на полигоне последовательно выполняя задания в автономном режиме за отведенное время.
\\\hspace*{1cm}
Задания на полигоне позволяют протестировать не столько механические характеристики робота (такие как проходимость), сколько его «интеллектуальные» возможности.
\\\hspace*{1cm}
Соревнования проходят в формате хакатона.

%%%
\subsection{Финансовое обеспечение} \hspace*{1cm}
Участие в соревновании бесплатное.

%%%
\subsection{Расписание}\hspace*{1cm}
В течение года проходит не менее пяти отборочных этапов и один финальный. Расписание этапов публикуется на официальном сайте Международных молодежных робототехнических соревнований «Кубок РТК».
\\\hspace*{1cm} В ходе сезона расписание может корректироваться (могут меняться или уточнятся даты проведения опубликованных этапов, добавляться новые этапы). 

%%%
\section{Требования к участникам} \hspace*{1cm}
Для участия в соревновании команды должны соответствовать следующим требованиям:
\begin{enumerate}[\textbullet]
	\item К участию приглашаются учащихся старших курсов лицеев и техникумов, студентов ВУЗов и СУЗов всех регионов РФ и стран СНГ;
	\item Возраст участников 15+;
	\item Количество участников в команде - от 1 до 3 человек;
	\item В каждой команде должен быть выбран капитан;
    \item Один участник может быть задействован только в одной команде в ходе текущих соревнований.
    \item Изменение состава команд запрещено в ходе очного этапа.  Запросы на изменение состава должны быть направлены организаторам не позднее, чем за 3 дня до начала очного этапа. 
    \item Каждая команда обязана использовать собственного робота.  Использование одного робота несколькими командами запрещено.
\end{enumerate}

%%%
\section{Требования к участникам} \hspace*{1cm}
\\\hspace*{1cm}
На соревнования команда представляет заранее собранного робота. Оснащение робота необходимым оборудованием для выполнения заданий на полигоне может быть выполнено как заранее, так и во время подготовки в ходе очного этапа. Конструкция робота и его оснащение (датчики, камеры и пр.) - произвольные, по желанию участника. 
\\\hspace*{1cm}
На полигоне не будет участков, требующих от робота повышенной проходимости, если это специально не оговорено в задании.
\\\hspace*{1cm}
Рекомендуемые габариты робота (в соответствии с габаритами заданий на полигоне) - не более (ДхШхВ) 400х300х350 мм.
\\\hspace*{1cm}
Максимальная масса робота - 10 кг. Робот должен иметь источник питания на борту и управляться либо автономно (вычислитель на борту робота), либо с использованием беспроводной связи для взаимодействия с вычислительным устройством, расположенном вне робота. Конструкция робота не должна причинять вред полигону и людям.

%%%
\section{Порядок проведения} \hspace*{1cm}
Соревнования состоят из подготовительного и очного этапов.

%%%
\subsection{Подготовительный этап} \hspace*{1cm}
Подготовительный этап начинается в день открытия регистрации на соревнование и заканчивается в момент начала очного этапа.
Не позднее дня открытия регистрации участникам становятся известны (публикуются на сайте соревнований «Кубок РТК» в разделе соответствующего хакатона):
    •  обобщенный перечень задач, которые потребуется решить на полигоне в автономном режиме, например, определение расстояния до объекта, распознавание контура, распознавание цвета, движение по линии;
    •  параметры некоторых объектов на полигоне, необходимые для проектирования робота, захватного устройства, размещения датчиков.
Используя эти данные, участники в ходе подготовительного этапа могут самостоятельно изучить и подготовить необходимые алгоритмы для выполнения заданий на соревновании, а также подготовить робота.

%%%
\subsection{Регистрация и отбор на очный этап} \hspace*{1cm}
Для участия в очном этапе необходимо на сайте соревнований «Кубок РТК» в разделе соответствующего хакатона заполнить регистрационную форму.
При регистрации участники должны отправить фотографию робота, а также имеют возможность отправить видео демонстрирующее работу робота и алгоритмов в рамках задач соответствующего этапа хакатона.
На очный этап проходят не более 12 команд, которые в случае необходимости будут отбираться судейской коллегией на основании указанных в регистрационной форме характеристик робота, а также предоставленных фото и видеоматериалов.

%%%
\subsection{Очный этап} \hspace*{1cm}
Длительность очного этапа не менеОбщее количество команд участников финала (квота) указывается на официальном сайте в разделе «Кубок РТК Высшая лига: Финал». В случае изменения квоты участников финала по независящим от организаторов причинам новые значения должны быть опубликованы на официальном сайте не позднее чем за 1 месяц до начала финальных соревнований.
В финал без отбора проходят команды-победители всех этапов сезона. Далее отбор финалистов в рамках квоты производится при помощи рейтинговой таблицы. Таблица формируется, публикуется и обновляется в ходе всего сезона на официальном сайте cup.rtc.ru в разделе «Кубок РТК Высшая лига». В ходе сезона все результаты всех команд вносятся в рейтинговую таблицу. Рейтинговая таблица формируется следующим образом: для каждой команды вычисляется среднее значение набранных баллов за все пройденные этапы. Полученные в результате расчета значения ранжируются от максимального к минимальному. Отбор команд в Финал соревнований «Кубок РТК Высшая лига» начинается с максимального значения.
От одной образовательной организации на финал может отобраться не более 
4 команд.
Если в ходе сезона в команде сменился второй или третий участник (не капитан) или команда сменила название, то команда не считается новой в рейтинге. Если участник уже зарегистрирован в рейтинговой таблице как капитан в одной из команд, то он не может регистрироваться в других этапах как второй участник в другой команде. Если участник уже зарегистрирован в рейтинговой таблице как второй член команды, то он может стать капитаном в новой команде, при этом он вычеркивается из состава предыдущей команды.е 2 дней.
На очном этапе участники получают подробное описание полигона и миссии с баллами за каждое задание. Также участники получают доступ к полигону для проведения тестовых заездов.
Участники должны самостоятельно подготовить и запрограммировать робота для выполнения заданий. По окончании времени на подготовку начинаются соревновательные попытки. По результатам попыток подводятся итоги и определяются победители.

%%%
\subsection{Подготовка робота в ходе очного этапа} \hspace*{1cm}
Первый день (или более, в случае, когда продолжительность очного этапа превышает 2 дня) - этап подготовки. Точная длительность этапа подготовки, а также время начала соревновательных попыток определяется судьями в день начала очного этапа (зависит от количества команд).
Участники самостоятельно собирают и программируют робота для выполнения заданий и миссии в целом. Каждая команда обязана иметь ноутбук с предустановленной ОС, который обеспечивается самими участниками команды, по собственному усмотрению, без ограничений. В ходе попытки робот должен выполнять задания самостоятельно, то есть быть автономным.
Программа, управляющая движением робота, должна быть создана непосредственно командой (участником) соревнований. В ходе подготовки участникам разрешается использование портативных носителей и сети Интернет.
ЗАПРЕЩАЕТСЯ взаимодействие с лицами, не являющимися членами команды, использование подсказок, в том числе с помощью электронных средств связи. В случае нарушения, команда по решению судьи будет дисквалифицирована. Участник может обращаться к судьям за разъяснениями правил соревнования и задания.
Участники могут выполнять тестовые заезды роботов на полигоне в любое время, кроме времени проведения попыток, карантина (см. далее), а также во время работы судей на полигоне.
Очередность тестовых заездов формируется командами самостоятельно в зависимости от готовности. В случае возникновения конфликта интересов между командами в выборе очередности судья может ограничить время на тестирование для каждой команды, а также количество подходов к полигону составив очередность тренировок.
В отведенное время между попытками, если робот не находится в карантине, участники имеют право на оперативное конструктивное и программное изменение робота (в том числе ремонт, замена элементов питания и прочее), если внесенные изменения не противоречат требованиям, предъявляемым к конструкции робота, и не нарушают регламент соревнований.

%%%
\section{Ход соревнований} \hspace*{1cm}
В завершающий день проходят непосредственно соревнования -  выполнение миссии на полигоне. Для каждой команды соревнования состоят как минимум из двух попыток (по решению судей количество попыток для всех команд может быть увеличено).

%%%
\subsection{Карантин} \hspace*{1cm}
За 15 минут до начала попыток все команды помещают роботов в карантин. В любой момент времени в карантине роботы должны быть выключены (при этом зарядка аккумуляторных батарей разрешена), конструкция робота должна оставаться неизменной. 
После помещения робота в карантин судья проводит жеребьевку, определяющую очередность выступления команд.

%%%
\subsection{Жеребьевка} \hspace*{1cm}
Порядок выступлений команд определяется случайной жеребьевкой. Каждой команде присваивается номер, определяющий её позицию в заранее установленном порядке выступлений. Жеребьевка продолжается до тех пор, пока все команды не получат номера. Если команда или её представитель отсутствуют, оставшиеся номера присваиваются им случайным образом.

%%%
\subsection{Подготовительное время} \hspace*{1cm}
С момента вызова команды запускается время на подготовку - 5 минут. Во время подготовительного времени команда имеет право использовать полигон для отладки, менять конструкцию робота или менять программный код.

%%%
\subsection{Ход попытки} \hspace*{1cm}
Попытка включает в себя запуск робота с ячейки Старт и движение робота по полигону до завершения выполнения миссии или до выезда за пределы полигона, либо застревания. Во время попытки на полигоне присутствует только один робот. На попытку отводится 5 минут. После запуска робота и до окончания попытки прикасаться к роботу запрещено, за исключением вмешательства. В ходе попытки возможно не более двух вмешательств.  Участникам команды разрешается вести фото и видеофиксацию попытки. В ходе попытки участники должны в ходе реального времени демонстрировать судьям, результаты выполнения заданий в понятном и читаемом виде. Так же обязательно к концу попытки должен быть сформирован отдельный файл (log-файл) содержащий все данные о распознавании объектов роботом, выполненных им действиях, а также метки времени выполнения действий должны записываться в и быть доступны к просмотру после окончания попытки в случае спорных ситуаций.

%%%
\subsection{Результаты попытки} \hspace*{1cm}
После завершения заезда необходимо освободить полигон и подписать протокол.  Протоколы, не подписанные участниками, недействительны.

%%%
\section{Результаты} \hspace*{1cm}
Выполнение каждого задания оценивается отдельно. Результат своей попытки можно узнать через 20 минут после ее окончания на стойке регистрации. В итоговый список баллов этапа идет лучшая из попыток.

%%%
\section{Судейство} \hspace*{1cm}
Контроль соревнований и подведение итогов осуществляется судейской коллегией в соответствии с регламентом соревнований.
Выполнение роботом заданий на полигоне и время окончания попытки фиксируются двумя судьями в протоколах. По окончании попытки капитан команды должен проверить судейский протокол и поставить подпись, соглашаясь с результатами. Протоколы без подписей не учитываются в итоговом списке баллов. Участники команды имеют право сфотографировать подписанный протокол.
Спорные моменты, возникающие в период соревнований, разрешаются судейской коллегией на месте. Апелляции принимаются в течение часа после окончания попытки.
Судья имеет право дисквалифицировать участника, чьи действия нарушают регламент по любой причине, включая, помимо прочего, неспортивное поведение как на площадке, так и за ее пределами в ходе проведения соревнований.

%%%
\section{Порядок определения победителя} \hspace*{1cm}
Побеждает команда, набравшая наибольшее количество баллов в итоговом списке. При наличии у двух команд одинакового количества баллов, побеждает команда, завершившая попытку за меньшее время. В случае, если время также одинаково, побеждает команда с наивысшим суммарным баллом по сумме всех попыток.

%%%
\subsection{Несколько независимых полигонов} \hspace*{1cm}
Если этап включает несколько независимых полигонов, формируется общий рейтинг, учитывающий результаты со всех полигонов и время прохождения.  Призовые места на данном этапе распределяются на основе этого общего рейтинга.  Выбор стратегии прохождения, оптимизирующей время, остается за участниками.

%%%
\section{Награждение} \hspace*{1cm}
Каждый зарегистрировавший очно участник получает диплом за участие. Команды-призеры (I, II и III места) награждаются дипломами и ценными призами. По окончании награждения результаты хакатона публикуются на сайте Международных молодежных робототехнических соревнований «Кубок РТК» в соответствующем соревнованию разделе, а также в группе соревнований «Вконтакте» и официальном телеграмм-канале).

%%%
\section{Отбор в финал} \hspace*{1cm}
Общее количество команд участников финала (квота) указывается на официальном сайте в разделе «Кубок РТК Высшая лига: Финал». В случае изменения квоты участников финала по независящим от организаторов причинам новые значения должны быть опубликованы на официальном сайте не позднее чем за 1 месяц до начала финальных соревнований.
В финал без отбора проходят команды-победители всех этапов сезона. Далее отбор финалистов в рамках квоты производится при помощи рейтинговой таблицы. Таблица формируется, публикуется и обновляется в ходе всего сезона на официальном сайте cup.rtc.ru в разделе «Кубок РТК Высшая лига». В ходе сезона все результаты всех команд вносятся в рейтинговую таблицу. Рейтинговая таблица формируется следующим образом: для каждой команды вычисляется среднее значение набранных баллов за все пройденные этапы. Полученные в результате расчета значения ранжируются от максимального к минимальному. Отбор команд в Финал соревнований «Кубок РТК Высшая лига» начинается с максимального значения.
От одной образовательной организации на финал может отобраться не более 
4 команд.
Если в ходе сезона в команде сменился второй или третий участник (не капитан) или команда сменила название, то команда не считается новой в рейтинге. Если участник уже зарегистрирован в рейтинговой таблице как капитан в одной из команд, то он не может регистрироваться в других этапах как второй участник в другой команде. Если участник уже зарегистрирован в рейтинговой таблице как второй член команды, то он может стать капитаном в новой команде, при этом он вычеркивается из состава предыдущей команды.

\newpage
\title{Регламент}
Всю информацию о проведении хакатонов можно найти на сайте Международных молодежных робототехнических соревнований «Кубок РТК» в разделе «Высшая Лига» https://cup.rtc.ru/khakaton


\end{document}