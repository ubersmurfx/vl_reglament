\pdfoutput=1
\documentclass[12pt]{article}
\usepackage[margin=2.5cm, top=1cm]{geometry}

% Required packages
\usepackage{amsmath,amsfonts,amsthm,amssymb,mathrsfs} % Mathematical typesetting and symbols
\usepackage{enumerate} % Custom enumeration labels
\usepackage{graphicx} % Figure inclusion
\usepackage{fancyhdr} % Перемещено сюда
\usepackage{hyperref} % Hyperlinks for citations, references, and URLs
\usepackage[numbers,sort&compress]{natbib} % Citation styling
\usepackage{subcaption} % Captions for subfigures
\usepackage{xcolor} % Text color
\usepackage[utf8x]{inputenc} % кодировка
\usepackage[english,russian]{babel} % добавление русского языка
\usepackage{cmap} % отображение юникод символов в pdf файле
\usepackage{eso-pic}
\usepackage{tikz}
\usepackage{ulem}
\usepackage{tabularx}
\usepackage{array} % Для выравнивания по правому краю
\usepackage{setspace}
\usepackage{xcolor}
% competition.tex
\usepackage{eso-pic}
\usepackage{xcolor}

\AddToShipoutPictureBG{%
    \ifnum\value{page}=1
        \put(4cm,26cm){\parbox[b][\paperheight]{\paperwidth}{%
        \fontsize{14}{14}\selectfont
        \color{gray!75}
        «Кубок РТК Высшая Лига: Финал, 26-28 ноября 2025 г.\par}}
    \fi
}

\usetikzlibrary{decorations.pathreplacing}

% Figure and table numbering by section
\counterwithin{figure}{section}
\counterwithin{table}{section}
\newenvironment{example}[1][]{\begin{ex}[#1]}{\hfill$\blacksquare$\end{ex}} % Add a black box to the end automatically, and accept an optional title argument
\newtheorem{ex}{Example}[section] % Examples are implemented as a type of Theorem

% Overriding abbrvnat BibTeX template to use the plainurl template's DOI hyperlinks
\newcommand*{\doi}[1]{\href{https://doi.org/#1}{\tt doi:~#1}}
\newcommand{\wrongPlod}{\color{red}-50}
\newcommand{\DetectPlod}{40}
\newcommand{\CatchPlod}{50}
\newcommand{\RottenPlod}{60}
\newcommand{\CatchRottenPlod}{100}
\newcommand{\StoredRottenPlod}{70}

%==============================================================================
\AddToShipoutPictureBG*{
    \ifnum\value{page}=1
        \put(2cm,26cm){\includegraphics[width=1.5cm]{figures/cup.jpg}}
    \fi
}

% background
\date{} % Добавлено для удаления даты
\title{Задание для очного этапа \\ Хакатона:«Сбор урожая»}

%==============================================================================
\begin{document}

\maketitle
\thispagestyle{empty}
\vspace*{-1cm}

\renewcommand{\arraystretch}{1}
\textbf{Задача} - автономный сбор урожая в условиях смоделированной тепличной среды.

Робот должен начать движение со стартовой ячейки, проехать по теплице и собрать заданные овощи или фрукты, 
а также утилизировать один гнилой фрукт или овощ.
Также по окончании обследования должен быть сформирован файл-протокол.

Для успешного выполенения задания необходимо: 
\begin{itemize}
    \item[>>] реализовать автономное движение по теплице;
    \item[>>] выполнить распознавание заданных овощей/фруктов;
    \item[>>] выполнить сбор урожая заданных овощей/фруктов;
    \item[>>] выполнить захват и доставку сгнившего фрукта/овоща в специальный контейнер.
\end{itemize}

Количество попыток - две. На подготовку к попытке дается 5 минут, на выполнение миссии - 5 минут. 
Очередность команд на попытку определяется жеребьевкой.  
В ходе выполнения миссии возможно только два вмешательства оператора.  
При вмешательстве робот устанавливается только на старт. 
В случае вмешательства баллы обнуляются. 
По требованию судьи участник обязан продемонстрировать наличие датчиков/камер на роботе и их работу!

\begin{itemize}
    \item[>>] Во время попытки баллы за каждое задание начисляются только один раз.
\end{itemize}

\vspace*{0.25cm}
\begin{tabularx}{\textwidth}{|X|c|p{2cm}|}
    \hline
    \textbf{Задание} & \textbf{Баллы} & \textbf{Результат}\\
    \hline
    Распознан заданный овощ/фрукт & \DetectPlod &  \\
    \hline
    Собран правильный овощ/фрукт & \CatchPlod & \\
    \hline
    Распознан сгнивший овощ/фрукт & \RottenPlod & \\
    \hline 
    Захвачен сгнивший овощ/фрукт & \CatchRottenPlod & \\
    \hline
    Перемещен сгнивший овощ/фрукт в контейнер & \StoredRottenPlod & \\
    \hline
    Вмешательство без штрафа (1) &  & \\
    \hline
    \textbf{Штрафы} & \textbf{Баллы} & \textbf{Результат} \\
    \hline
    Собран неправильный овощ/фрукт & \wrongPlod &  \\
    \hline
\end{tabularx}
\vspace*{0.25cm}

\textcolor{red}{ВАЖНО!} В ходе попытки необходимо в режиме реального времени демонстрировать судьям, 
результаты выполнения заданий в понятном и читаемом виде. 
Так же обязательно к концу попытки должен быть сформирован log-файл с 
указанием результатов распознавания и соответствующего момента времени.

\end{document}