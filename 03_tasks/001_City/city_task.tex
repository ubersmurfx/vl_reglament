\pdfoutput=1
\documentclass[12pt]{article}
\usepackage[margin=2.5cm, top=1cm]{geometry}

% Required packages
\usepackage{amsmath,amsfonts,amsthm,amssymb} % Mathematical typesetting and symbols
\usepackage{enumerate} % Custom enumeration labels
\usepackage{graphicx} % Figure inclusion
\usepackage{fancyhdr} % Перемещено сюда
\usepackage{hyperref} % Hyperlinks for citations, references, and URLs
\usepackage[numbers,sort&compress]{natbib} % Citation styling
\usepackage{subcaption} % Captions for subfigures
\usepackage{xcolor} % Text color
\usepackage[utf8x]{inputenc} % кодировка
\usepackage[english,russian]{babel} % добавление русского языка
\usepackage{cmap} % отображение юникод символов в pdf файле
\usepackage{eso-pic}
\usepackage{tikz}
\usepackage{ulem}
\usepackage{tabularx}
\usepackage{array} % Для выравнивания по правому краю
\usepackage{setspace}
\usepackage{xcolor}
% competition.tex
\usepackage{eso-pic}
\usepackage{xcolor}

\AddToShipoutPictureBG{%
    \ifnum\value{page}=1
        \put(4cm,26cm){\parbox[b][\paperheight]{\paperwidth}{%
        \fontsize{14}{14}\selectfont
        \color{gray!75}
        «Кубок РТК Высшая Лига: Финал, 26-28 ноября 2025 г.\par}}
    \fi
}

\usetikzlibrary{decorations.pathreplacing}

% Figure and table numbering by section
\counterwithin{figure}{section}
\counterwithin{table}{section}
\newenvironment{example}[1][]{\begin{ex}[#1]}{\hfill$\blacksquare$\end{ex}} % Add a black box to the end automatically, and accept an optional title argument
\newtheorem{ex}{Example}[section] % Examples are implemented as a type of Theorem

% Overriding abbrvnat BibTeX template to use the plainurl template's DOI hyperlinks
\newcommand*{\doi}[1]{\href{https://doi.org/#1}{\tt doi:~#1}}
\newcommand{\penaltyBusPoints}{\color{red}-50}
\newcommand{\boardingPassangersFirstZone}{115}
\newcommand{\boardingPassangersSecondZone}{135}
\newcommand{\passangersDelievery}{160}
\newcommand{\complianceTrafficRules}{90}
%==============================================================================
\AddToShipoutPictureBG*{
    \ifnum\value{page}=1
        \put(2cm,26cm){\includegraphics[width=1.5cm]{figures/cup.jpg}}
    \fi
}

% background
\date{} % Добавлено для удаления даты
\title{Задание для очного этапа \\ Хакатона:«Движение по городу»}

%==============================================================================
\begin{document}

\maketitle
\thispagestyle{empty}
\vspace*{-2cm}

\renewcommand{\arraystretch}{1}
\textbf{Задача} - автономная доставка условных пассажиров из зон посадки в зону высадки.

Задание считается выполненным если робот проехал по городу соблюдая правила дорожного движения, остановился на 2 секунды 
у одной или двух зон посадки перед знаком «Остановка» (для условной посадки пассажиров) после 
чего доехал до зоны высадки и остановился у нее перед знаком «Парковка». Очередность проезда зон посадки значения не имеет, 
также не обязательно проезжать обе зоны посадкию.
За соблюдение дорожной разметки (движение по своей полосе) начисляются дополнительные баллы.

Для успешного выполенения задания необходимо: 
\begin{itemize}
    \item[>>] реализовать автономное движение по городу;
    \item[>>] выполнить распознавание знаков дорожного движения.
\end{itemize}

Количество попыток - две. На подготовку к попытке дается 5 минут, на выполнение миссии - 5 минут. 
Очередность команд на попытку определяется жеребьевкой.  
В ходе выполнения миссии возможно только два вмешательства оператора.  
При вмешательстве робот устанавливается только на старт. 
В случае вмешательства баллы обнуляются. 
По требованию судьи участник обязан продемонстрировать наличие датчиков/камер на роботе и их работу!

\begin{itemize}
    \item[>>] Во время попытки баллы за каждое задание начисляются только один раз.
    \item[>>] Перед началом попыток расположение знаков будет изменено.
    \item[>>] Знаки дорожного движения расположены случайным образом.
    \item[>>] Маршрут движения на полигоне нефиксированный и формируется участни-
ком исходя из правил дорожного движения.
    \item[>>] Движение по полосам засчитывается только в том случаем, если робот доехал до парковки (знак P) и забрал хотя бы одних пассажиров.
    \item[>>] Положение на полигоне общественного транспорта перед началом попыток
определяется случайным образом и остается неизменным в ходе попыток.
\end{itemize}

\vspace*{0.25cm}
\begin{tabularx}{\textwidth}{|X|c|p{2cm}|}
    \hline
    \textbf{Задание} & \textbf{Баллы} & \textbf{Результат}\\
    \hline
    Выполнена посадка первой группы пассажиров & \boardingPassangersFirstZone &  \\
    \hline
    Выполнена посадка второй группы пассажиров & \boardingPassangersSecondZone & \\
    \hline
    Пассажиры доставлены в зону высадки & \passangersDelievery & \\
    \hline 
    Соблюдено правило движения только по полосам & \complianceTrafficRules & \\
    \hline
    \textbf{Штрафы} & \textbf{Баллы} & \textbf{Результат} \\
    \hline
    Столкновение с общественным транспортом & \penaltyBusPoints &  \\
    \hline
\end{tabularx}
\vspace*{0.25cm}

\end{document}