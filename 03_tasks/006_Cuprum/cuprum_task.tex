\pdfoutput=1
\documentclass[12pt]{article}
\usepackage[margin=2.5cm, top=1cm]{geometry}

% Required packages
\usepackage{amsmath,amsfonts,amsthm,amssymb,mathrsfs} % Mathematical typesetting and symbols
\usepackage{enumerate} % Custom enumeration labels
\usepackage{graphicx} % Figure inclusion
\usepackage{fancyhdr} % Перемещено сюда
\usepackage{hyperref} % Hyperlinks for citations, references, and URLs
\usepackage[numbers,sort&compress]{natbib} % Citation styling
\usepackage{subcaption} % Captions for subfigures
\usepackage{xcolor} % Text color
\usepackage[utf8x]{inputenc} % кодировка
\usepackage[english,russian]{babel} % добавление русского языка
\usepackage{cmap} % отображение юникод символов в pdf файле
\usepackage{eso-pic}
\usepackage{tikz}
\usepackage{ulem}
\usepackage{tabularx}
\usepackage{array} % Для выравнивания по правому краю
\usepackage{setspace}
\usepackage{xcolor}
% competition.tex
\usepackage{eso-pic}
\usepackage{xcolor}

\AddToShipoutPictureBG{%
    \ifnum\value{page}=1
        \put(4cm,26cm){\parbox[b][\paperheight]{\paperwidth}{%
        \fontsize{14}{14}\selectfont
        \color{gray!75}
        «Кубок РТК Высшая Лига: Финал, 26-28 ноября 2025 г.\par}}
    \fi
}

\usetikzlibrary{decorations.pathreplacing}

% Figure and table numbering by section
\counterwithin{figure}{section}
\counterwithin{table}{section}
\newenvironment{example}[1][]{\begin{ex}[#1]}{\hfill$\blacksquare$\end{ex}} % Add a black box to the end automatically, and accept an optional title argument
\newtheorem{ex}{Example}[section] % Examples are implemented as a type of Theorem

% Overriding abbrvnat BibTeX template to use the plainurl template's DOI hyperlinks
\newcommand*{\doi}[1]{\href{https://doi.org/#1}{\tt doi:~#1}}
\newcommand{\WrongButton}{\color{red}-20}
\newcommand{\firstKZ}{150}
\newcommand{\secondKZ}{130}
\newcommand{\thirdKZ}{100}
\newcommand{\writeButtonsecondKZ}{40}
\newcommand{\writeButtonfirstKZ}{40}
\newcommand{\writeButtonthirdKZ}{40}

%==============================================================================
\AddToShipoutPictureBG*{
    \ifnum\value{page}=1
        \put(2cm,26cm){\includegraphics[width=1.5cm]{figures/cup.jpg}}
    \fi
}

% background
\date{} % Добавлено для удаления даты
\title{Задание для очного этапа \\ Хакатона:«Электролиз меди»}

%==============================================================================
\begin{document}

\maketitle
\thispagestyle{empty}
\vspace*{-1cm}

\renewcommand{\arraystretch}{1}
\textbf{Задача} - разработать автономного робота для поиска коротких замыканий между электродами по наличию магнитного поля.

Робот должен начать движение со стартовой ячейки, проехать по 8-ми ваннам, определить, по наличию магнитного поля, пары анод-катод 
с коротким замыканием и вернувшись в стартовую ячейку нажать кнопки с номерами ванн, где обнаружено замыкание.
Также по окончании обследования должен быть сформирован файл-протокол.

Для успешного выполенения задания необходимо: 
\begin{itemize}
    \item[>>] реализовать автономное движение по электролитическим ваннам;
    \item[>>] выполнить поиск коротких замыканий.
\end{itemize}

Количество попыток - две. На подготовку к попытке дается 5 минут, на выполнение миссии - 5 минут. 
Очередность команд на попытку определяется жеребьевкой.  
В ходе выполнения миссии возможно только два вмешательства оператора.  
При вмешательстве робот устанавливается только на старт. 
В случае вмешательства баллы обнуляются. 
По требованию судьи участник обязан продемонстрировать наличие датчиков/камер на роботе и их работу!

\begin{itemize}
    \item[>>] Во время попытки баллы за каждое задание начисляются только один раз.
    \item[>>] Короткие замыкания определяются безконтактным способом.
    \item[>>] Баллы за правильное нажатие кнопок могут быть начислены только в том случае, если было обнаружено короткое замыкание.
\end{itemize}

\vspace*{0.25cm}
\begin{tabularx}{\textwidth}{|X|c|p{2cm}|}
    \hline
    \textbf{Задание} & \textbf{Баллы} & \textbf{Результат}\\
    \hline
    Обнаружено короткое замыкание №1 & \firstKZ &  \\
    \hline
    Обнаружено короткое замыкание №2 & \secondKZ & \\
    \hline
    Обнаружено короткое замыкание №3 & \thirdKZ & \\
    \hline
    Нажата кнопка короткого замыкания №1 & \writeButtonfirstKZ & \\
    \hline
    Нажата кнопка короткого замыкания №2 & \writeButtonsecondKZ & \\
    \hline
    Нажата кнопка короткого замыкания №3 & \writeButtonthirdKZ & \\
    \hline
    \textbf{Штрафы} & \textbf{Баллы} & \textbf{Результат} \\
    \hline
    Кнопка нажата неверно (максимум 3 раза за попытку) & \WrongButton &  \\
    \hline
\end{tabularx}
\vspace*{0.5cm}

\textcolor{red}{ВАЖНО!} В ходе попытки необходимо в режиме реального времени демонстрировать судьям, 
результаты выполнения заданий в понятном и читаемом виде. 
Так же обязательно к концу попытки должен быть сформирован log-файл с 
указанием результатов распознавания и соответствующего момента времени.

\end{document}