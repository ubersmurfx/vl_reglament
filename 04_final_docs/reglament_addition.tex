\pdfoutput=1
\documentclass[12pt]{article}
\usepackage[margin=2.5cm, top=1cm]{geometry}

% Required packages
\usepackage{amsmath,amsfonts,amsthm,amssymb,mathrsfs}
\usepackage{enumerate}
\usepackage{graphicx}
\usepackage{fancyhdr}
\usepackage{hyperref}
\usepackage[numbers,sort&compress]{natbib}
\usepackage{subcaption}
\usepackage{xcolor}
\usepackage[utf8x]{inputenc}
\usepackage[english,russian]{babel}
\usepackage{cmap}
\usepackage{eso-pic}
\usepackage{tikz}
\usepackage{ulem}
\usepackage{tabularx}
\usepackage{array}
\usepackage{setspace}
\usepackage{xcolor}
\usepackage{mathtools}
% competition.tex
\usepackage{eso-pic}
\usepackage{xcolor}

\AddToShipoutPictureBG{%
    \ifnum\value{page}=1
        \put(4cm,26cm){\parbox[b][\paperheight]{\paperwidth}{%
        \fontsize{14}{14}\selectfont
        \color{gray!75}
        «Кубок РТК Высшая Лига: Финал, 26-28 ноября 2025 г.\par}}
    \fi
}

\usetikzlibrary{decorations.pathreplacing}

% Figure and table numbering by section
\counterwithin{figure}{section}
\counterwithin{table}{section}
\newenvironment{example}[1][]{\begin{ex}[#1]}{\hfill$\blacksquare$\end{ex}} % Add a black box to the end automatically, and accept an optional title argument
\newtheorem{ex}{Example}[section] % Examples are implemented as a type of Theorem

% Overriding abbrvnat BibTeX template to use the plainurl template's DOI hyperlinks
\newcommand*{\doi}[1]{\href{https://doi.org/#1}{\tt doi:~#1}}
\newcommand{\penaltyBusPoints}{\color{red}-50}
\newcommand{\boardingPassangersFirstZone}{115}
\newcommand{\boardingPassangersSecondZone}{135}
\newcommand{\passangersDelievery}{160}
\newcommand{\complianceTrafficRules}{90}
\renewcommand{\arraystretch}{1}
%==============================================================================
\AddToShipoutPictureBG*{
    \ifnum\value{page}=1
        \put(2cm,26cm){\includegraphics[width=1.5cm]{figures/cup.jpg}}
    \fi
}

% background
\date{} % Добавлено для удаления даты
\title{Дополнение к регламенту финала соревнований «Кубок РТК Высшая Лига»}

%==============================================================================
\begin{document}


\begin{center}
\vspace*{10cm} % Отступ сверху для вертикального центрирования
{\LARGE Дополнение к регламенту финала соревнований «Кубок РТК Высшая Лига»}
\end{center}


\thispagestyle{empty}
\newpage
\thispagestyle{plain}


\vspace*{-0.5cm}
\section*{Полигоны}
Финальный этап соревнований включает в себя четыре различных полигона для автономного прохождения.

\begin{enumerate}
    \item \textbf{«Сбор Урожая»}
    \begin{itemize}
        \item Положение на полигоне центрального ящика для гнилых фруктов и овощей определяется случайным образом
        \item Ящик обозначен двумя цифровыми метками
        \item Для распознавания представлен полный набор овощей и фруктов, в том числе и гнилых
    \end{itemize}
    
    \item \textbf{«Движение по городу»}
    \begin{itemize}
        \item Знаки дорожного движения расположены таким образом, что маршрут движения робота на 
        полигоне нефиксированный и формируется участником исходя из правил до­рожного движения
        \item Маршрут движения формируется случайным образом исходя из правил дорожного движения
        \item Положение на полигоне общественного транспорта определяется случайным образом
    \end{itemize}
    
    \item \textbf{«Медный завод»}
    \begin{itemize}
        \item Номера ванн на полигоне остаются неизменными на протяжении всех соревнований и обозначены цифрами
        \item Нумерация ванн будет обозначена в первый день соревнований
        \item Нумерация кнопок определяется случайным образом
    \end{itemize}
    
    \item \textbf{«Ледяной вызов»}
    \begin{itemize}
        \item Положение сугробов и арматур определяется случайнм образом
    \end{itemize}
\end{enumerate}

\vspace*{-0.5cm}
\section*{Карантин}
\begin{itemize}
    \item Карантин вводится за 10 минут до старта попыток
    \item Робот должен находиться в карантине с полностью отключенным питанием, за исключением процедуры зарядки аккумуляторов.
    \item Любые модификации аппаратной части или программного кода (включая жёстко прописанные параметры - хардкод) 
    запрещены и являются основанием для дисквалификации
    \item Запрещается любым способом препятствовать или мешать выступлениям других команд
\end{itemize}

\vspace*{-0.5cm}
\section*{Ход попытки}
\begin{itemize}
    \item На выполнение задания на одном полигоне предоставляется одна попытка.
    \item На выступление одной команды на одном полигоне отводится \textbf{15 минут}:
    \begin{itemize}
        \item 5 минут подготовка
        \item 10 минут попытка
    \end{itemize}
    \item \textcolor{red}{Любое изменение кода запрещено}
    \item Каждая команда имеет \textbf{3 вмешательства} за одну попытку
    \item При запуске решения ноутбуки и/или оборудование для демонстрации выполнения задания должны находиться на столе организаторов и контролироваться судьями
    \item В зачет идет лучший результат из 4х возможных запусков робота на полигоне (с учетом вмешательств).
\end{itemize}

\vspace*{-0.5cm}
\section*{Бонус за время}
\begin{itemize}
    \item Максимальный бонус за время рассчитывается по формуле:
    \begin{center}
    \fbox{%
        \begin{minipage}{0.8\textwidth}
        \centering
        \vspace{8pt}
        $
        \text{Бонус за время B} = T \times \dfrac{S}{500} \times \dfrac{1}{6}
        $
        \smallskip
        \small
        \begin{tabular}{@{}c@{}}
        где: $T$ — время выполнения (сек)
        $S$ — набранные баллы за лучший заезд
        \end{tabular}
        \vspace{8pt}
        \end{minipage}%
    }
    \end{center}
    
    \item Бонус за время ${B}$ \textbf{не превышает 100 баллов}, что составляет \textbf{20\%} от максимального количества баллов полигона
    \item Максимальное количество баллов за первый и второй выбранный полигон составляет
    \begin{itemize}
        \item Первый полигон: $R_1 = 500 + \text{бонус за время}$ (макс. 600 баллов)
        \item Второй полигон: $R_2 = 500 + \text{бонус за время}$ (макс. 600 баллов)
    \end{itemize}
\end{itemize}

\vspace*{-0.5cm}
\section*{Итоговая оценка}
Итоговое распределение мест среди команд определяется по фомруле
\begin{center}
\fbox{%
    \begin{minipage}{0.8\textwidth}
    \centering
    \vspace{8pt}
    $
    \text{Итоговый балл} = R_1 + R_2
    $
    
    \smallskip
    \small
    где: $R_1$ — баллы за первый выбранный полигон, $R_2$ — баллы за второй выбранный полигон
    \vspace{8pt}
    \end{minipage}%
}
\end{center}

\begin{itemize}
    \item Максимальное возможное количество баллов: \textcolor{red}{\textbf{1200}}
\end{itemize}

\end{document}